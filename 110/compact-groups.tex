\section{Overview}
\subsection{Next Few Weeks}
Over the next few weeks, we will cover

\begin{enumerate}[1.]
\item compact groups,
\item basics of algebraic groups,
\item Flag varieties and Bruhat decomposition,
\item Highest weight theory and Borel-Weil theorem,
\item Tannakian categories.
\end{enumerate}

The next unit will start with the smooth representations of reductive
groups over local fields.

\subsection{Course Overview}

Our goal is to get to the geometric Satake equivalence. Let $k$ be an
algebraically closed field. Let $G$ be a connected reductive algebraic group
over $k$ (e.g. $\gl n k$, $\sl n k$, $\sp {2n} k$, $G_2$). The geometric Satake
equivalence says there is the equivalence of abelian tensor (!) categories
\[ \perv_{``G(k[[t]])"}\left(``\quotient{G(k((t)))}{G(k[[t]])}"\right)
\isomorphism \rep(\langdual G). \]
The left-hand side is the category of perverse sheaves on $\gr G$,
the affine Grassmannian of $G$, and the right-hand the category of finite
dimensional algebraic representations of the Langlands dual group of $G$
(e.g if $G = \gl n k$, $\langdual G = G$, and if $G = \sp {2n} k$,
$\langdual G = \so {2n + 1} k$).

\subsubsection{Part I of the Class} Understand $\rep(\langdual G)$.

\subsubsection{Part II of the Class} The classical Satake isomorphism and
smooth representations of reductive groups over local fields. It is the
isomorphism
\[ \mathcal{C}^\infty_{c, ``G(k[[t]])"}\left(``\quotient{G(k((t)))}{G(k[[t]])}"
  \right) \longiso R(\langdual G), \]
obtained by taking the Grothendieck group of both sides (applying $\ggroup$)
of the geometric Satake equivalence. The left-hand side is called the
spherical Hecke algebra of $G$, which arises from studying the smooth
representations of $G(k((t)))$ or $G(\Qp)$, and is made into an algebra by
convolution of functions. The right-hand side is the representation ring of
the Langlands dual of $G$.

\subsubsection{Part III of the Class} Understand the $\perv$ of the geometric
Satake equivalence. This entails understanding the six operations (e.g. $f_*,
f^*$) on derived categories of (constructible) sheaves on algebraic varieties,
and the full subcategory of perverse sheaves.

\subsubsection{Part IV of the Class} This will be distributed throughout the
course, but the goal is to understand what $\gr G$ is.

\subsubsection{Part V of the Class} The geometric Satake equivalence.

\section{Representation and Structure Theory of Compact Lie Groups}

\begin{defn}
A \textbf{Lie group} is a group object in the category of smooth manifolds, i.e.
$m: G \times G \rightarrow G, \ \mathrm{inv}: G \rightarrow G,
\ \mathrm{id}: {1} \rightarrow G$, satisfying the expected commutative diagrams.
\end{defn}

\begin{exmpl}
The following are Lie groups.

\begin{itemize}
\item $\gl n \reals \subset M_n(\reals) \iso \reals^{n^2}$
\item $\sp {2n} \reals = \{ g \in \sl {2n} \reals \ | \
  g \ \textrm{preserves a non-degenerate alternating form} \}$
\item More generally, closed subgroups $G \xhookrightarrow{} \gl n \reals$.
\item Unipotent matrices, invertible upper triangular matrices (Borel subgroup).
\item $\sl 2 \reals$. Remark: $\pi_1(\sl 2 \reals) \iso \ints$. This can be
established by looking at the Iwasawa decomposition of $\sl 2 \reals$.
\item $\widetilde{\sl 2 \reals}$, the metaplectic cover of $\sl 2 \reals$ is a
Lie group but can't be embedded into $\gl n \reals$.
\end{itemize}
\end{exmpl}

The last two are less relevant for us.

\subsection{Lie Algebras}
The Lie algebra $\mathfrak{g}$ of $G$ is given by the tangent space at the 
identity. It is a priori a vector space, but the Lie bracket gives it an algebra
structure, which we will describe briefly.

There is a map $\Psi: G \rightarrow \aut(G)$ since $G$ acts on $G$ by 
conjugation. Since $\Psi(g)$ preserves the identity, we can look at the map
\[ \operatorname{Ad}: G \rightarrow \aut_\reals(\mathfrak{g}), \]
given by \[ g \rightarrow \restr{d\Psi(g)}{1}.\] By differentiating at the 
identity, we obtain the map
\[ \operatorname{ad}: \mathfrak{g} \rightarrow \into_\reals(\mathfrak{g}). \]
Then we can define the Lie bracket as $[X, Y] := \operatorname{ad}(X)(Y)$.

\begin{exmpl}
For $G = \gl n \reals$, $\mathfrak{g} = M_n(\reals)$, and $[X, Y] = XY - YX$.
\end{exmpl}

\subsection{Exponential Map}

Here we will explain the role of Lie algebras, by constructing a map
$\op{exp} : \mathfrak{g} \rightarrow G$. For $G = \gl n k$,
$\op{exp}(X) = \sum_n X^n / n!$.

\subsubsection{$\op{exp}$ Facts} Before giving the construction, note several
facts.

\begin{enumerate}[1.]
\item $\op{exp}$ is a local diffeomorphism near 0, by the implicit function
theorem.
\item $\op{exp}$ is not a homomorphism once $G$ is non-abelian.
\end{enumerate}

\subsubsection{$\op{exp}$ Construction}
Take $X \in \mathfrak{g}$. Use left-multiplication to propagate $X$ to a
left-invariant vector field $\xi_X$ on $G$, i.e. $\xi_X(g) = (m_g)_*(X).$

Construct an integral curve  $\varphi_X$ for this vector field by the following 
procedure. Solve the  differential equation
$\psi(0) = 1, \psi_X'(t)
= \xi_X(\psi_X(t))$. This gives a 
function $\psi_X : (-\varepsilon, \varepsilon) \rightarrow G$.

Note that $\psi_X$ is a homomorphism. Fix $s$ and let $t$ vary. Let
$\alpha(t) = \psi_X(s) \psi_X(t)$ and let $\beta(t) = \psi_X(s + t)$. Then
$\alpha(0) = \beta(0)$, and $\alpha'(t) = \xi_X(\alpha(t))$ and
$\beta'(t) = \xi_X(\beta(t))$ by left-invariance of $\xi_X$. Thus,
$\alpha = \beta$.

This can be extended to a homomorphism $\varphi_X : \reals \rightarrow G$,
by setting $\varphi_X(t) = \phi_X(t/N)^N$, where $N$ is an integer such that
$t/N < \varepsilon/2$. This does not depend on the choice of $N$; if $M$ was
another such integer, we would have
\[ \psi_X\left(\dfrac{t}{MN}\right)^N = \psi_X\left(\dfrac{t}{M}\right), 
  \psi_X\left(\dfrac{t}{MN}\right)^M = \psi_X\left(\dfrac{t}{N}\right). \]
Hence \[ \psi_X\left(\dfrac{t}{M}\right)^M = \phi_X(\dfrac{t}{MN})^{MN}
 = \phi_X\left(\dfrac{t}{N}\right)^N. \]
Using this definition and the fact that $\varphi_X$ is a homomorphism in
$(-\varepsilon, \varepsilon)$, it can be shown $\varphi_X$ is a homomorphism
$\reals \rightarrow G$.

\begin{defn}
$\op{exp}_G(X) := \varphi_X(1)$.
\end{defn}

\begin{rmk}
For $G$ a compact connected Lie group, $\mathfrak{g}$ its Lie algebra,
$\op{exp}_G: \mathfrak{g} \rightarrow G$ is surjective. The basic idea of the
proof is to use the Killing form on a Lie algebra. Use multiplication to 
transport it to a Riemannian metric. The integral curves are geodesics, and any
two points are connected by a geodesic.
\end{rmk}

\begin{ex}
Find a $G$ where $\op{exp}_G$ is not surjective.

We do not need to look far for such an example. If $G = \gl 2 \reals$, and
$\lambda_1, \lambda_2$ are two \emph{distinct} negative real numbers, consider
\[B = \left(\begin{matrix}
  \lambda_1 & 0 \\
  0 & \lambda_2
\end{matrix}\right). \]
If there is an $A \in M_n(\reals)$ with $\op{exp}(A) = B$, and $A$ has
eigenvalues $\alpha_1, \alpha_2$, then $e^{\alpha_i} = \lambda_i$. Since
$\lambda_i$ are negative, $\alpha_1$ and $\alpha_2$ are complex conjugates. Thus
$\lambda_1, \lambda_2$ have the same absolute value, a contradiction.
\end{ex}

\subsection{Representation of Compact Lie Groups}

\begin{defn}
A representation of $G$ a compact Lie group is a continuous (hence smooth)
homomorphism $G \rightarrow \gl n \cmplx$.
\end{defn}

The motto here is to study representations of $G$ by restricting to a maximal
tori.

\begin{defn}
A (compact) torus is a (compact) connected abelian Lie group.
\end{defn}

\begin{exmpl}
$S^1$ is a torus, and almost all examples are isomorphic to $(S^1)^n$. If $T$ 
is a  torus, its Lie algebra $\mathfrak{t}$ is isomorphic to $\reals^n$ and 
surjects onto $T$ via $\op{exp}_T$. The kernel is a lattice $\ints^r$, so
$T \iso (S^1)^{r} \times \reals^{n - r}$. For compact tori, $r = n$.
\end{exmpl}

\begin{defn}
A \textbf{character} (or \textbf{weight}) of $T$ is a continuous homomorphism
$T \rightarrow S^1$. Write $\cgroup T$ for the abelian group of characters of
$T$.
\end{defn}

\begin{ex}
Prove the following lemma.
\begin{lem}
If $T \iso \reals^n/\ints^n$, then all characters of $T$ have the form
\[ (x_1, ..., x_n) \rightarrow \prod_{r = 1}^n e^{2 \pi i a_r x_r}, \]
where all $a_r \in \ints$.
\end{lem}
\begin{proof}
We just need to know what homomorphisms there are from $\reals/\ints$ to $S^1$ 
(viewed as the unit circle in $\cmplx$). This is the same as knowing the 
homomorphisms from $\reals/\ints \rightarrow \reals/\ints$, and any such
$\varphi$ lifts to a homomorphism $\phi$ from $\reals$ to $\reals$, by using 
twice the fact that a universal covering space covers any connected cover.

This construction shows $\phi(1) \equiv 0 \pmod{1}$, so
for some integer $n$, $\phi(1) = n$, $\phi(a) = na$,
$b \phi(a/b) = \phi(a) = na$. By continuity,
$\phi(x) = nx$ for any $x$. Then $\varphi(x) = \phi(x) = nx \pmod{1}$. As
$x \rightarrow e^{2 \pi i x}$ is a homeomorphism $\reals/\ints \rightarrow S^1$,
any homomorphism $\reals/\ints \rightarrow S^1$ is given by
$x \rightarrow e^{2\pi i n x}$, for some $n$.
\end{proof}
\end{ex}

\begin{lem}
Any compact torus has a (dense set of) topological generators, i.e. there is
$t \in T$ with $T = \overline{\langle t \rangle}.$
\end{lem}

\begin{proof}
Assume $T \iso \reals^n/\ints^n$, and take $T \ni t = (t_1, ..., t_n)$. Let
$H := \overline{\langle t \rangle}$. Then use the

\begin{lem}
$H = T$ if and only if $1, t_1, ..., t_n$ are $\rats$-linearly independent.
\end{lem}

If $H \neq T$, then $T/H$, a compact torus, has a non-trivial character $\chi$,
and \[ \chi(x_1, ..., x_n) = \prod_r e^{2 \pi i a_r x_r}, \]  for some integers
$a_r$ not all zero. Then $\restr{\chi}{H} = 1$ if and only if $\chi(t) = 1$
if and only if $\sum a_r t_r \in \ints$, which implies $1, t_1, ..., t_n$ are
$\rats$-linearly dependent.
\end{proof}

\begin{thm}
Let $G$ be a compact connected Lie group. Fix a maximal torus $T \subset G$.
\begin{enumerate}[(1)]
\item\label{conju} Every $g \in G$ is conjugate to an element of $T$.
\item Any other maximal torus $T' \subset G$ is $G$-conjugate to $T$.
\item $C_G(T) = T$.
\end{enumerate}
\end{thm}

\begin{ex}
Find a counter-example for non-compact groups. We already know \ref{conju}
for $U(n)$. Why?

An example of a non-compact Lie group without any tori is $\reals^n$. A maximal 
torus of $U(n)$ is the subgroup of all diagonal matrices. The fact that any 
unitary matrix can be diagonalized by another unitary matrix gives \ref{conju} 
for $U(n)$.
\end{ex}

\begin{proof} A sketch of the proof:

\begin{enumerate}[(1)]
\item\label{surj} follows from surjectivity of $\op{exp}_G$.
\item is immediate from \ref{surj}.
\item uses surjectivity of $\op{exp}_G$.
\end{enumerate}

\end{proof}

\subsection{Onto Representation Theory}
Let $G$ be a compact connected group as before.

\begin{prop}
Any finite dimensional representation of $G$ is isomorphic to a direct sum of
irreducible representations.
\end{prop}

\begin{proof}
The proof is the same as in the finite case. Let $V \in \rep(G)$. Fix a 
hermitian inner product $( \ , \ )$ on $V$. Produce a $G$-invariant inner product $\langle \ , \ \rangle$ by using the Haar measure that exists on $G$:
\[ \langle v, w \rangle = \int_G (g \cdot v, g \cdot w) \ dg. \]
If $W$ is a $G$-subrepresentation of $V$, then $V$ decomposes as
$V = W \oplus W^\perp$ ($W^\perp$ is $G$-stable since $\langle \ , \ \rangle$ is
$G$-invariant.
\end{proof}

The following is due to Schur.

\begin{cor}
If $V$ is irreducible, then $\into_{\cmplx[G]}(V) = \cmplx$.
\end{cor}

\begin{proof}
Let $T \in \into_{\cmplx[G]}(V)$. Since $V$ is finite-dimensional, $T$ has an
eigenvalue $\lambda$. Then if $T$ is non-constant, $T - \lambda \cdot 1$ acts 
on $V$ and satisfies $0 \neq \ker(T - \lambda \cdot 1) \subset V$. Since $V$ is
irreducible, $\ker(T -\lambda \cdot 1) = V$, so $T = \lambda$.
\end{proof}

\begin{defn}
Let $V \in \rep(G)$. The \textbf{character} $\chi_V$ of $V$ is a homomorphism
$G \rightarrow \cmplx$ given by $g \rightarrow \tr(\restr{g}{V})$.
\end{defn}

Some basic facts about characters.

\begin{enumerate}[1.]
\item $\chi_V$ only depends on the isomorphism class of $V$.
\item $\chi_V$ only depends on the conjugacy class of its argument.
\item $\chi_{V \oplus W} = \chi_V + \chi_W$.
\item $\chi_{V \otimes W} = \chi_V \chi_W$.
\item $\chi_{V^*} = \conj{\chi_V}$, where $V = \operatorname{Hom}(V, \cmplx)$.
\end{enumerate}

The first four properties give us a homomorphism
$K_0(\rep(G)) = R(G) \rightarrow C(G)$. Viewing $\chi_V$ in $L^2(G)$ with an
inner product
\[ \langle f_1, f_2 \rangle = \int_G \conj{f_1(g)} f_2(g) \ dg, \]
we have the following basic calculation.

\begin{prop}
\begin{enumerate}[(1)]
\item For $V, W \in \rep(G)$,
$\langle \chi_V, \chi_W \rangle = \dim \operatorname{Hom}_G (V, W).$
\item Characters of irreducible representations are orthnormal (Schur).
\item $\chi_V = \chi_W$ if and only if $V \iso W$.
\end{enumerate}
\end{prop}
