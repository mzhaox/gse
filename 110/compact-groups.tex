\section{Overview}
\subsection{Next Few Weeks}
Over the next few weeks, we will cover

\begin{enumerate}[1.]
\item compact groups,
\item basics of algebraic groups,
\item Flag varieties and Bruhat decomposition,
\item Highest weight theory and Borel-Weil theorem,
\item Tannakian categories.
\end{enumerate}

The next unit will start with the smooth representations of reductive
groups over local fields.

\subsection{Course Overview}

Our goal is to get to the geometric Satake equivalence. Let $k$ be an
algebraically closed field. Let $G$ be a connected reductive algebraic group
over $k$ (e.g. $\gl n k$, $\sl n k$, $\sp {2n} k$, $G_2$). The geometric Satake
equivalence says there is the equivalence of abelian tensor (!) categories
\[ \perv_{``G(k[[t]])"}\left(``\quotient{G(k((t)))}{G(k[[t]])}"\right)
\isomorphism \rep(\langdual G). \]
The left-hand side is the category of perverse sheaves on $\gr G$,
the affine Grassmannian of $G$, and the right-hand the category of finite
dimensional algebraic representations of the Langlands dual group of $G$
(e.g if $G = \gl n k$, $\langdual G = G$, and if $G = \sp {2n} k$,
$\langdual G = \so {2n + 1} k$).

\subsubsection{Part I of the Class} Understand $\rep(\langdual G)$.

\subsubsection{Part II of the Class} The classical Satake isomorphism and
smooth representations of reductive groups over local fields. It is the
isomorphism
\[ \mathcal{C}^\infty_{c, ``G(k[[t]])"}\left(``\quotient{G(k((t)))}{G(k[[t]])}"
  \right) \longiso R(\langdual G), \]
obtained by taking the Grothendieck group of both sides (applying $\ggroup$)
of the geometric Satake equivalence. The left-hand side is called the
spherical Hecke algebra of $G$, which arises from studying the smooth
representations of $G(k((t)))$ or $G(\Qp)$, and is made into an algebra by
convolution of functions. The right-hand side is the representation ring of
the Langlands dual of $G$.

\subsubsection{Part III of the Class} Understand the $\perv$ of the geometric
Satake equivalence. This entails understanding the six operations (e.g. $f_*,
f^*$) on derived categories of (constructible) sheaves on algebraic varieties,
and the full subcategory of perverse sheaves.

\subsubsection{Part IV of the Class} This will be distributed throughout the
course, but the goal is to understand what $\gr G$ is.

\subsubsection{Part V of the Class} The geometric Satake equivalence.

\section{Representation and Structure Theory of Compact Lie Groups}

\begin{defn}
A \textbf{Lie group} is a group object in the category of smooth manifolds, i.e.
$m: G \times G \rightarrow G, \ \mathrm{inv}: G \rightarrow G,
\ \mathrm{id}: {1} \rightarrow G$, satisfying the following diagrams.

%TODO: commutative diagrams
\end{defn}

\begin{exmpl}
The following are Lie groups.

\begin{itemize}
\item $\gl n \reals \subset M_n(\reals) \iso \reals^{n^2}$
\item $\sp {2n} \reals = \{ g \in \sl {2n} \reals \ | \
  g \ \textrm{preserves a non-degenerate alternating form} \}$
\item More generally, closed subgroups $G \xhookrightarrow{} \gl n \reals$.
\item Unipotent matriceslatex, invertible upper triangular matrices (Borel subgroup).
\item $\sl 2 \reals$, which has $\pi_1(\sl 2 \reals) \iso \ints$.
%TODO(hint): SO(2) \xhookrightarrow{} SL_2(R) 
\item $\widetilde{\sl 2 \reals}$, the metaplectic cover of $\sl 2 \reals$ is a
Lie group but can't be embedded into $\gl n \reals$.
\end{itemize}

The last two are less relevant for us.

\subsection{Basic Structure of Lie Groups}
\subsubsection{Lie Algebras}
The Lie algebra $\mathfrak{g}$ of $G$ is given by the tangent space at the 
identity. It is a priori a vector space, but the Lie bracket gives it an algebra
structure, which we will describe briefly.

There is a map $G \rightarrow \aut(G)$ since $G$ acts on $G$ by conjugation.
This induces $\operatorname{Ad}: G \rightarrow \aut_\reals (\mathfrak{g})$.




\end{exmpl}
